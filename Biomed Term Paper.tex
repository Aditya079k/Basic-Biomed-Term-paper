\documentclass[12pt]{article}
\usepackage{graphicx}
\usepackage[document]{ragged2e}
\usepackage[utf8]{inputenc}

\begin{document}
\begin{titlepage}
\begin{center}

\large{\bfseries PROJECT REPORT}\\
[1mm]
\large{On}\\
[1cm]
\Large{\textit{\bfseries DIAGNOSIS AND MONITORING}}\\
[0.65cm]
\large{\textit{Submitted by:}}\\
{Aditya Shrotriya}\\
{Roll No.: 21111005}\\
{1st Semester, Biomedical Engineering}\\
[0.5cm]
\large{NATIONAL INSTITUTE OF TECHNOLOGY, RAIPUR}\\
[0.2cm]

\begin{figure}[h]
\centering
\includegraphics[scale=.6]{NITRaipur_Logo}\\
[0.5cm]
\end{figure}

\large{\textit{Under the supervision of:}}\\
[0.25cm]
\large{Dr. Saurabh Gupta}\\
[2.5mm]
\large{National Institute of Technology, Raipur}

\end{center}
\end{titlepage}
\clearpage
\begin{center}
\Large{\bfseries Acknowledgement}\\
[1cm]
\end{center}
\large{I am grateful to Dr. Saurabh Gupta, for his proficient supervision of the term project on "Diagnosis and Monitoring". I am very thankful to you sir for your guidance and support.}\\
[1.5cm]

\begin{flushright}
\large{Aditya Shrotriya}\\
[0.5cm]
\large{21111005}\\
[0.5cm]
\large{1st Semester, Biomedical Engineering}\\
[0.5cm]
\large{National Institute of Technology, Raipur}\\
[1.5cm]
\end{flushright}
\large{Date of Submission: 8th April 2022}
\clearpage

\begin{center}
\Large{\bfseries ABSTRACT}\\
[.5cm]
\end{center}

\begin{center}
{\bfseries Biomedical Engineering} is the rising field in medical science by involving the knowledge of biology and medicine in combination with the principals of engineering to develop devices and procedures which can solve the greatest number of medical and health-related  problems in this modern world. One such solution provide by Biomedical Studies is Diagnosis and Monitoring. The aim of this project report is to  provide a standard information about {\bfseries Diagnosis and Monitoring} and the role of {\bfseries Biomedical Engineering} applications towards it by expressing its availability and its results which are up to date. The innovation also establishes how successful does it takes part in medical science and yet research to be done for unsuccessful cases.
\end{center}
\clearpage

\section{Diagnosis}

Diagnosis is the identification of the nature and cause of a certain phenomenon. Diagnosis is used in many different disciplines, with variations in the use of logic, analytics, and experience, to determine "cause and effect". In systems engineering and computer science, it is typically used to determine the causes of symptoms, mitigations, and solutions.

\subsection{AI in Diagnosis}

As a sub-field in artificial intelligence, Diagnosis is concerned with the development of algorithms and techniques that are able to determine whether the behaviour of a system is correct.

\begin{figure}[ht]
\centering
\includegraphics[scale=0.4]{AI diagnosis}
\caption{AI in Diagnosis}
\end{figure} 

If the system is not functioning correctly, the algorithm should be able to determine, as accurately as possible, which part of the system is failing, and which kind of fault it is facing. The computation is based on observations, which provide information on the current behaviour.

\subsection{Diagnosability}

A system is said to be diagnosable if whatever the behavior of the system, we will be able to determine without ambiguity a unique diagnosis.

The problem of diagnosability is very important when designing a system because on one hand one may want to reduce the number of sensors to reduce the cost, and on the other hand one may want to increase the number of sensors to increase the probability of detecting a faulty behavior.

Several algorithms for dealing with these problems exist. One class of algorithms answers the question whether a system is diagnosable; another class looks for sets of sensors that make the system diagnosable, and optionally comply to criteria such as cost optimization.

The diagnosability of a system is generally computed from the model of the system. In applications using model-based diagnosis, such a model is already present and doesn't need to be built from scratch.

\subsection{Computer-Aided Diagnosis}

Computer-aided detection, also called computer-aided diagnosis, are systems that assist doctors in the interpretation of medical images. Imaging techniques in X-ray, MRI, and ultrasound diagnostics yield a great deal of information that the radiologist or other medical professional has to analyze and evaluate comprehensively in a short time. CAD systems process digital images for typical appearances and to highlight conspicuous sections, such as possible diseases, in order to offer input to support a decision taken by the professional.

\begin{figure}[ht]
\centering
\includegraphics[scale=.4]{Computer_aided_diagnosis}
\caption{X-Ray of hand for Diagnosis by Computer Software}
\end{figure}

CAD is an interdisciplinary technology combining elements of artificial intelligence and computer vision with radiological and pathology image processing. A typical application is the detection of a tumor. For instance, some hospitals use CAD to support preventive medical check-ups in mammography (diagnosis of breast cancer), the detection of polyps in the colon, and lung cancer.

\section{Monitoring}

In medicine, monitoring is the observation of a disease, condition or one or several medical parameters over time.

It can be performed by continuously measuring certain parameters by using a medical monitor (for example, by continuously measuring vital signs by a bedside monitor), and/or by repeatedly performing medical tests (such as blood glucose monitoring with a glucose meter in people with diabetes mellitus).

\begin{figure}[ht]
\centering
\includegraphics[scale=.5]{Medical_monitor}
\caption{Display Device of Medical Monitor}
\end{figure}

Transmitting data from a monitor to a distant monitoring station is known as telemetry or biotelemetry.

\subsection{Medical Monitoring}

A medical monitor or physiological monitor is a medical device used for monitoring. It can consist of one or more sensors, processing components, display devices (which are sometimes in themselves called "monitors"), as well as communication links for displaying or recording the results elsewhere through a monitoring network.

An entirely new scope is opened with mobile carried monitors, even such in sub-skin carriage. This class of monitors delivers information gathered in body-area networking (BAN) to e.g. smart phones.

\subsection{Techniques in developing}

As {\bfseries biomedical research}, nanotechnology and nutrigenomics advances, realizing the human body's self-healing capabilities and the growing awareness of the limitations of medical intervention by chemical drugs-only approach of old school medical treatment, new researches that shows the enormous damage medications can cause, researchers are working to fulfill the need for a comprehensive further study and personal continuous clinical monitoring of health conditions while keeping legacy medical intervention as a last resort.

\section{Conclusion}

By this report we can conclude that {\bfseries biomedical engineering} field does have a major role in this developing world and the applications are well designed in order to proceed without any harm. When science and technology combine to enhance the procedures  for {\bfseries diagnosis  and monitoring} which is an  advantage for  the health care  system mainly  for the cancer  patients. Nevertheless, the research for certain health issues is still under study for a better outcome in the future.
\clearpage

\end{document}